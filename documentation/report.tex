\documentclass[11pt, letterpaper]{article}
\usepackage[utf8]{inputenc}
\title{Medical Assistent}
\author{Adrian Locher, Jason Benz}
\date{\today}

\usepackage{graphicx}

\begin{document}
\pagenumbering{gobble}
\maketitle

\newpage
\tableofcontents
\newpage

\pagenumbering{arabic}

\section{Einleitung}
    \subsection{Motivation}
        \paragraph{}
            Die Frage, ob ein Arzt besucht werden soll, ist häufig ein Dilemma.
            Einerseits, will man nicht diejenige Person sein, welche bei unbedenklichen Symptomen
            überreagiert, andererseits könnten sich die Symptome ja verschlimmern und vielleicht
            hätte der Arztbesuch das Problem frühzeitig beheben können. Aus solchen Gründen ist man ja
            schliesslich versichert.

        \paragraph{}
            Eine ganz andere Beudeutung hat dieses Dilemma ausserdem seit dem Anfang
            der Covid-19 Situation. Ärzte und anderes Gesundheitspersonal, gelten als besonders
            stark ausgelastet. Auf der kehrseite hingegen, sollte man bei Covid-19-artigen Symptomen
            auch nicht zögern und sich testen lassen. Welche Symptome nun genau als "Covid-Symptome"
            gelten und welche nicht, ist teilweise sehr unübersichtlich, was die lage auch nicht
            einfacher werden lässt.

        \paragraph{}
            Das Ziel unseres Projekts, soll sein eine lösung zu entwickeln, mit der unnötige Arztbesuche
            generell, aber besonders in Situationen wie jetzt vermieden werden können.
            Das Ziel ist nicht, dass einfach grundsätzlich weniger Menschen Arztbesuche machen, sondern
            vielmehr, in Grenzfällen bei der Entscheidung zu helfen.
        
        \paragraph{}
            Um dabei auch wirklich das Gesundheitspersonal maximal zu entlasten, besteht unser
            Lösungsansatz nicht etwa aus einer Hotline. Ziel ist es ein AI gestützes System zu entwickeln,
            welches ganz ohne Menschliche Interaktion auf seiten der Gesundheitsbranche auskommt.

\newpage

\section{Java Client}

\newpage

\section{Backend}

\newpage

\section{Discussion}


\end{document} 